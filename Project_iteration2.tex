\documentclass[paper=a4, english, ngerman, romanian]{scrartcl}

\usepackage[a4paper,left=2cm,right=2cm,top=2.5cm,bottom=3cm]{geometry}
\usepackage[ngerman]{babel}
\usepackage{tabularx}
\usepackage[utf8]{inputenc}
\usepackage{multirow}
\usepackage{listings}
\usepackage{graphicx}
\usepackage[absolute]{textpos}
\usepackage{hyperref}

\parindent 0pt
\lstset{basicstyle={\ttfamily\scriptsize}, tabsize=4}
\begin{document}

\begin{titlepage}
	\title{Datenbanksysteme SS17: Projekt}	
	\subtitle{Dozentin: Agnes Voisard}
	\author{Bernadeta Chisarau, Dor Cohen, Mihai Renea}
	\date{\normalsize \today}
\end{titlepage}

\maketitle								% Erstellt das Titelblatt
\vspace*{-8cm}							% rückt Logo an den oberen Seitenrand
\makebox[\dimexpr\textwidth+1cm][r]{	%rechtsbündig und geht rechts 1cm über Layout hinaus
	\includegraphics[width=0.4\textwidth]{src/fu_logo} % fügt FU-Logo ein
}

\vspace{7cm}							% Abstand
\rule{\linewidth}{0.8pt}				% horizontale Linie
	
Unser Repository mit allen Dateien ist unter: \url{https://github.com/derMihai/DBS_Project} zu finden 
	
	\section{Aufgabe: Datenbankschema erstellen}
	\lstinputlisting[language=sql, caption={}]{election_schema.sql} 
	
	\section{Aufgabe: Datenbereinigung}
	\lstinputlisting[language=java, caption={}]{CSV_Cleaner.java} 

	\section{Aufgabe: Datenimport}
	\lstinputlisting[language=java, caption={}]{Data_importer.java} 	
	
	\subsection{CSV Parser}
	\lstinputlisting[language=java, caption={}]{CSV_Parser.java}
	\lstinputlisting[language=java, caption={}]{Tweet.java} 	
 	\section{Aufgabe: Webserver}
 	Wir haben ins File \lstinline[basicstyle=\ttfamily\small]|/etc/apache2/ports.conf| die Zeile  \lstinline[basicstyle=\ttfamily\small]|Listen 8050| hinzugefügt, und im File \ \\ \lstinline[basicstyle=\ttfamily\small]|/etc/apache2/sites-enabled/test.conf| das Folgende reingeschrieben:
	\begin{lstlisting}
<VirtualHost *:8050>
        ServerAdmin webmaster@localhost
        DocumentRoot /var/www/test
</VirtualHost>
<Directory /var/www/test>
        Deny from all
        Allow from 127.0.0.1 ::1
        Allow from localhost
</Directory>
	\end{lstlisting}
	Weiter das gegebene \lstinline[basicstyle=\ttfamily\small]|index.html| File wurde mit einem Java-Script ergänzt und in \lstinline[basicstyle=\ttfamily\small]|/var/www/test| kopiert. Nach einem Apache-Server Restart ist diese Page unter Port 8050 erreichbar.

	Inhalt der \lstinline[basicstyle=\ttfamily\small]|index.html| File:
	\begin{lstlisting}
<!doctype html>
<html>
    <head>
       <meta charset="utf-8">
       <title>Eine Webseite</title>
    </head>
    <body>
       <label for="eingabe">
          Ihr Name:
          <input id="feld" name="eingabe" />
       </label>
       <button id="knopf" type="button" onclick="hallo()">
           Klick mich!
       </button>
       <div id="bereich"></div>
       <script>
           function hallo(){
		      var input = document.getElementById("feld").value;
              if(input == "") input = "Unbekannter";      
              document.getElementById("bereich").innerHTML = "Hallo " + input + "!";
           }
       </script>
    </body>
</html>

	\end{lstlisting}
	
%	\begin{center}
%	\includegraphics[scale=0.5]{Screenshot_2017-05-02_11-34-51}
%	\end{center}

\end{document}
