\documentclass[paper=a4, english, ngerman, romanian]{scrartcl}

\usepackage[a4paper,left=2cm,right=2cm,top=2.5cm,bottom=3cm]{geometry}
\usepackage[ngerman]{babel}
\usepackage{tabularx}
\usepackage[utf8]{inputenc}
\usepackage{multirow}
\usepackage{listings}
\usepackage{graphicx}
\usepackage[absolute]{textpos}
\usepackage{amsmath}
\usepackage{mathtools}
\usepackage{amssymb}
\usepackage{dsfont}
\usepackage{wasysym}
\usepackage{enumitem}
\usepackage{stmaryrd}

\parindent 0pt
\lstset{basicstyle={\ttfamily\scriptsize}, tabsize=4}
\begin{document}

\begin{titlepage}
	\title{Datenbanksysteme SS17: Projekt\\ 3. Iteration}	
	\subtitle{Dozentin: Agnes Voisard}
	\author{Bernadeta Chișărău, Dor Cohen, Mihai Renea}
	\date{\normalsize \today}
\end{titlepage}

\maketitle								% Erstellt das Titelblatt
\vspace*{-8cm}							% rückt Logo an den oberen Seitenrand
\makebox[\dimexpr\textwidth+1cm][r]{	%rechtsbündig und geht rechts 1cm über Layout hinaus
	\includegraphics[width=0.4\textwidth]{src/fu_logo} % fügt FU-Logo ein
}

\vspace{7cm}							% Abstand
\rule{\linewidth}{0.8pt}				% horizontale Linie
	\section{Clusteranalyse}
		Für die Clusteranalyse haben wir den K-Means Algorithmus mithilfe der java-ml Library eingesetzt. Der Algorithmus partitioniert die Menge der Hashtags in 6 Clusters, wo jedes Hashtag ein 2-dimensionales Vektor mit den folgenden Metriken ist:
		\begin{itemize}
			\item Hashtag-Wichtigkeit -- als Durchschnitt der Wichtigkeitswerten aller Tweets, die Hashtag $h$ enthalten.\\
			Wichtigkeit $W_T(t)$ eines Tweets $t$:\\
			\begin{equation*}
				W_T(t) = ^4\sqrt{\frac{t_{favorites\ count} + t_{retweet\ count}}{2}}
			\end{equation*}
			Wichtigkeit $W_H(h)$ eines Hashtags h:
			\begin{equation*}
				W_H(h) = \frac{\sum_{t \in T} W_T(t)}{|T|}
			\end{equation*}
			Wo $T$ die Menge der Tweets, die Hashtag $h$ enthalten.
			
			\item Hashtag-Occurence -- wie oft ein Hashtag insgesamt auftaucht.
		\end{itemize}
		
		Anschließend speichern wir die neu-erzeugten Informationen in einer neuen Tabelle, \textit{hashtag}. Damit kann man für die Visualisierung die gebrauchten Werte einfach ablesen. Dadurch entsteht die aktuelle DB-Schema:
		
		\begin{center}
			\includegraphics[scale=0.6]{src/MinMax_Diagram}
		\end{center}
		
		\pagebreak

	\section{Datenvisualisierung}
	
		Um die Datenvisualisierung zu vereinfachen haben wir ein Programm geschrieben (Node\_data\_creator.java), das die Informationen aus der Datenbank in JSON-Dateien bereitstellt, und zwar:
			\begin{itemize}
				\item	\textit{plots.json} - Informationen für die Visualisierung des Hashtagnetzwerkes - Knotenpositionen (nodes-Array) und Verbindungen (edges-Array):
				\begin{lstlisting}
{
   "nodes": [
     {
       "color": "rgb(r,g,b)",
       "size": 100,
       "x": "x",
       "y": "y",
       "id": "hname",
       "label": "hname",
       "type": "tweetegy"
     },
     .
     .
     .
   ],
   "edges": [
     {
       "id": "i",
       "source": "hname1",
       "target": "hname2"
     },
     .
     .
     .
    ]
}
				\end{lstlisting}
				
				\item \textit{days.json} - Liste aller Tagen, mit der Anzahl der verschiedenen Hashtags, für die Zeitanalyse:
				
				\begin{lstlisting}
[
   {
     "x": x,
     "y": sum over "htags",
     "label": yy-mm-dd,
     "htags": [
       {
         "y": y1,
         "hname": "hname1"
       },
       {
         "y": y2,
         "hname": "hname2"
       },
       .
       .
       .
     ]
   },
   .
   .
   .
]
				\end{lstlisting}
			\end{itemize}
	
\end{document}
